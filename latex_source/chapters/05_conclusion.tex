% !TeX root = ../main.tex
% Add the above to each chapter to make compiling the PDF easier in some editors.

\chapter{Conclusion}\label{chapter:conclusion}
Reversi is an amazing game to learn game playing artificial. The primary reason being the small branching factor, allowing the computer to look ahead in abundance, thrashing human intuition and reasoning. This leads to an extending gap in the Othello-playing ability of humans and computers. After implementing only a few basic heuristics, me as the developer stood no chance against my own creation. That was an amazing feeling. \newline It was a lot of fun implementing the game in a C - a new language for me. Bitboards as a data structure and being concerned about memory leaks were also new to me. My favorit part was adding personal code to improve performance. There is a lot which I would like to add the project with more time. For example, a better calculation of the heuristic weights. That point hunts me the most. I know that the AI could have probably been even better with slight adjustments to the heuristic weights, but no opponent I implemented could help improve the values. \newline Also to this report I would have liked to add more segments, but it is already too long. Just to name a few:
\begin{enumerate}
	\item The additional data structures like game\_stage enumerations or global and internal bitboards.
	\item The tweak to the minimax algorithm to give winning a higher value, but dependent on how much depth is left an even higher value.
	\item Adding graphical visualization to the heuristic algorithm explanation chapter.
\end{enumerate}
All in all I had a really good time on this project and learned a lot for my future as a developer.

